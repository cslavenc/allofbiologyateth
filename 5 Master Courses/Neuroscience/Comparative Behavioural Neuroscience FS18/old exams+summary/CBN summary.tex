\documentclass[twosided, a4paper, pt11]{article}
\usepackage{enumitem}
\usepackage{a4wide}


\begin{document}
	\title{Comparative Behavioural Neuroscience - Summary (2018)}
	% maybe add more information here
	
	\maketitle
	
	\tableofcontents
	\newpage
	
	\section{PART 1 - Animal Behaviour and Learning}
	
	\subsection{Different types of learning in animal models}
	\subsubsection{Pavlovian learning - S-R learning}
	For an US, an animal will give an UR. A neutral stimulus can be paired with an US, such that the neutral stimulus will eventually be enough to induce a response. The neutral stimulus is the called a CS. The CS elicits the CR. The CR need not be the same as the UR.
	\paragraph{Examples:} A dog salivates when he sees food (US). He starts to salivate (UR). This can be paired with a tone (neutral stimulus $\rightarrow$ CS), where the dog also starts to salivate eventually (CR=UR).\\
	When mice are given a foot-shock (US) they start running around (UR). Paired with a sound (CS), the will actually show freezing behaviour (CR), before the foot-shock occurs. Here, it is necessary that the tone occurs in close temporal proximity as the shock.
	\paragraph{Conditioned inhibition:} Negative relationship between US and CS. In the above example, the CS predicted the occurrence of the US. Likewise, the CS can predict the absence of US. In \textbf{conditioned inhibition} the CS (e.g. tone) predicts that there will not be a foot-shock for some time, thus the CR is no freezing behaviour.
	$\Rightarrow$ Contiguity or co-occurrence is \textbf{not necessary} to produce Pavlovian learning.
	
	
	\paragraph{Wagner suprise model and Kamin Block effect.}
	The Wagner suprise model provides us with a mathematical framework on how fast an animal can learn. A novel stimulus will be learned quickly (if it is beneficial or aversive). Repetition of the same stimulus will establish the memory, but the learning will be smaller than before.\\
	In \textbf{extinction learning}, the CS-CR is unpaired (the animal learns CS does not follow CR) when the US does not occur. Example: After a tone, there is no foot-shock anymore, thus the freezing behaviour (CR) will slowly be extinct according to the Wagner surprise model.\\
	Formula: $\Delta$V = $\alpha$ * ($\lambda$ - V), with $\Delta$V change in strength for trial, $\alpha$ $\in$ [0,1], $\lambda$ maximal CS-US association strength, V CS-US strength.\newline
	
	In the Kamin block effect, CS1 is paired with US, which elicits a CR1. A second CS2 is paired with CS1 that also elicits CR1. If the CS2 alone is presented, there will be no response, since the association has been learned with CS1.\\
	$\Rightarrow$ Contiguity or co-occurrence is \textbf{not sufficient} for Pavlovian learning to occur.
	
	\subsection{Goal-oriented learning}
	
	\subsubsection{Operant conditioning}
	Operant conditioning involves more action from the animal. An initially neutral stimulus will lead to a reward (UR), which reinforces the neutral stimulus positively or negatively: Operant Stimulus $\rightarrow$ UR $\rightarrow$ reinforcer $\Rightarrow$ Stimulus $\rightarrow$ CR. An initially neutral stimulus can be pressing a lever for example (which results into a reward or a punishment).
	
	\subsubsection{Outcome-response and Response-outcome learning}
	Both theories describe goal-oriented behaviour. In OR-theory, the outcome precedes the actions, thus the actions are chosen accordingly. In RO-theory, the animal gains an overview of several possibilities (responses it can perform) and therefore, it has to simulate the outcomes and opt for the most attractive outcome. These theories are based on the belief and desire criterion and on instrumentality (= operant conditioning).
	
	\paragraph{OR-theory.} In the mind, an animal wants a certain outcome to realize. Therefore, it chooses such a response that the outcome will take place $\Rightarrow$ outcome (e.g. reward) precedes action.
	
	\paragraph{RO-theory.} The animal has to think about different responses and their outcomes (e.g. when in a T-maze) in order to get a reward. Thus, it gains an overview of responses and evaluates these in accordance to its beliefs and desires.
	
	\subsubsection{Pavlovian instrumental transfer: PIT = SOR}
	In Pavlovian-to-instrumental transfer, the animal first undergoes Pavlovian conditioning such that a CS is linked to a US resulting into a CR (e.g. CS = light, CR = positive response). Also, the animal undergoes OR conditioning (operant conditioning basically) such that it gains an intentional behaviour. Now, upon the presentation of the Pavlovian CS (here: light), the animal will perform the OR conditioned actions in order to obtain its reward.\\
	specific PIT: CS$_{pav}$ = CS$_{instr}$ in transfer (neural substrate: NAcc shell and BLA)\\
	general PIT: CS$_{pav}$ != CS$_{instr}$ in transfer (neural substrate: NAcc core and CNA)\\
	
	\subsection{The anatomy of the memory system: MTL}
	Components of the medial temporal lobe (=: MTL): hippocampal regions (DG, CA3, CA1, subiculum) and parahippocampal regions (LEA and MEA of the entorhinal cortex, perirhinal cortex, parahippocampal cortex). The neocortex projects to the parahippocampal regions which further project to the subregions of the hippocampus starting with the DG. CA1 and subiculum backproject to parahippocampal regions which backproject to the neocortex.
	
	\subsection{Pathway of the MTL}
	NEOCORTEX $\rightarrow$ parahippocampal cortex $\rightarrow$ MEA (where-pathway) $\rightarrow$ DG and CA3 $\rightarrow$ CA1 $\rightarrow$ subiculum; CA1 and subiculum $\rightarrow$ NEOCORTEX\\
	NEOCORTEX $\rightarrow$ perirhinal cortex $\rightarrow$ LEA (what-pathway) $\rightarrow$ DG and CA3 $\rightarrow$ CA1 $\rightarrow$ subiculum; CA1 and subiculum $\rightarrow$ NEOCORTEX\newline
	
	MEA and LEA both project to the same neural populations in DG and CA3, but these project to distinct populations in CA1 and subiculum (different "what and where" neurons in CA1 and subiculum).\newline
	
	\subsection{Functions of the MTL}
	The MTL is mainly conserved with What- and Where-information. The perirhinal cortex identifies the what-information of an object $\rightarrow$ LEA of entorhinal cortex (also involved in object recognition). The parahippocampal cortex receives processed spatial information of sensory content $\rightarrow$ MEA of entorhinal cortex (identifies in which context a memory has been presented previously - where-information). These project to the hippocampus, where MEA and LEA inputs (and when-information) are integrated. The \textbf{perforant pathway} connects MEA to the hippocampus. Lesioning the perforant pathway results into deficits concerning where-memory. The hippocampus is crucial in storing long-term spatial memory.\newline
	
	\section{PART 2 - Emotional Information Processing}
	
	\subsection{Rewarding stimuli}
	Important brain structures: VTA, NAcc, VP, mPFC, AMYG\newline
	
	\textbf{Dopamine synthesis:} L-DOPA is processed to dopamine via the Dopa-decarboxylase in the VTA. There, VMATs take dopamine up and are stored at the presynpatic side, waiting to be released upon an action potential.\newline
	
	VTA and NAcc are involved in motivation, liking and wanting behaviour of an animal. Notably, dopamine plays a central role in learning the relationship between a CS and US. Dopamine signalling will most of the time be increased to expecting a CS in the VTA and NAcc, while not so to a US. In depression, dopamine signalling is compromised and so are wanting and liking as well as CS-US relationship learning.
	
	\paragraph{Reward prediction error.} The reward prediction error (=: RPE) is the difference of the expectancy of the animal to a reward with the actual consumation of that reward. In brains, it is realized by the firing of dopamine neurons (and GABA neurons). Dopamine neurons have a constant baseline firing. Furthermore, GABA neurons inhibit these dopamine neurons, but in total, the firing of dopamine neurons is stronger. When an animal is exposed to a surprisingly rewarding stimulus, the baseline firing of the dopamine neurons will experience a sudden pulse in dopamine firing (increase in firing), while if a reward is not delivered, the firing will be less than baseline due to GABA neurons overpowering dopamine signalling. It then follows: RPE is positive if the reward was unexpected or better than expected. RPE is neutral (baseline) if the reward was as expected. RPE is negative if the reward is not delivered or less satisfactory than initially believed.\\
	Furthermore, RPE can also be applied to aversive stimuli which were not delivered $\Rightarrow$ this results into a relief and an increase in dopamine firing.\\
	
	RPE is often studied in the VTA and NAcc, although any structure possessing DA receptors can be studied such as the PFC (although the dopamine pulses occur at a slower time scale $\Rightarrow$ it has been proposed that this is due to the PFC searching for more complex rules in reality that lead to the reward).
	
	\subsubsection{Specific cases: Dopamine and opioids in NAcc}
	Stimulation of NAcc with dopamine leads to:
	\begin{itemize}
	\item NAcc - VP increase in firing related to a CS predicting reward.
	\item Increased VP firing does not increase US liking.
	\item Dopamine stimulation only increases the craving for CS, but not for US in general.
	\end{itemize}

	Stimulation of NAcc with opioids:
	\begin{itemize}
	\item VP firing increase to CS predicting a reward.
	\item VP firing increase to US predicting a reward.
	\end{itemize}
	
	\subsection{Aversive stimuli}
	\textbf{Anatomy of amygdala:} The amygdala has three nuclei called lateral amygdala (LA), basolateral amygdala (BLA) and central nucleus of the amygdala. The LA receives sensory input which it transmits to the BLA. There, the computation and processing of these stimuli mainly occurs. The central nucleus projects the final output to other brain areas.\\
	The amygdala occurs on both hemispheres and it is posterior to the hippocampus.\\
	The LA is involved in the process of fear conditioning (freezing in mice as CR to a CS).\newline
	
	Fear conditioning increases CS-evoked spike firing in the LA. This increase is associated to LTP-like changes on the postsynaptic synapse. Fear conditioning, as any learning process, increases the number of AMPARs on the postsynaptic cell. This results into a faster activation of the Ca$^{2+}$-induced signalling cascade (the postsynaptic cell is faster depolarized, such that Mg$^{2+}$ blockage of NMDARs is removed faster).\newline
	
	Blocking the CRF receptors on the central nucleus will inhibit fear learning. Furthermore, \textbf{oxytocin} (a neuropeptide\footnote{Oxytocin is produced in the PVN of the hypothalamus and is released by the posterior pituitary gland.}) will inhibit the activity of the central nucleus of the amygdala when binding to the oxytocin receptors.
	%\textbf{Anatomy of hippocampus:}
	
	\subsection{MR and GR in emotional learning and stress}
	The stress response occurs via the HPA axis in the brain. The PVN produces CRF which enters the corticotrophic cells of the pituitary gland. There, ACTH is released into the blood stream. ACTH binds to the adrenal gland which finally releases corticosteroids\footnote{Cortisol occurs in humans, while corticosteroids occur in rodents. The receptors are the same for both types though.} (NE, cortisol, corticosterone etc.). The corticosteroids can easily pass the membrane, since they are steroids, and bind to intracellular receptors. Corticosteroids send a negative inhibition signal to the pituitary gland, such that release of ACTH is stopped and to the PVN, such that the transcription of CRF genes is stopped.\\
	Corticosteroids have an enhancing effect on memory when emotional arousal is involved $\Leftrightarrow$ the information is novel $\Leftrightarrow$ the information is \textbf{not} neutral! This increases emotional memory consolidation. On the other hand, high amounts of stress (= glucocorticoids) impair working memory (mPFC) and memory retrieval.
	
	MR and GR are both receptors for corticosteroids. MR only occurs in the amygdala and has a ten times higher affinity(K$_{D,MR}$ = 0.5 nM) than GR for corticosteroids. GR occurs globally. In fear learning, MR mainly signals, but under high stress conditions, corticosteroids also bind to GR and induce its signalling cascade.\\
	MR and GR both occur in the BLA and CNA.
	\begin{itemize}
		\item MR antagonism $\Rightarrow$ no fear learning (= no consolidation) if antagonist administered before learning task (= blocking of contextual learning).
		\item MR anatognism administered after learning task $\Rightarrow$ fear learning occurred almost as in controls $\Rightarrow$ MR needed for learning, but not for consolidation
		\item GR antagonism $\Rightarrow$ no effect on learning or short-term, but GR antagonism inhibits long-term memory of fear learning (= no phenotype in fear chamber after 24h - as if it has completely forgotten the previous day).
	\end{itemize}

	\subsection{Interaction between NE and Corticosteroids}
	Adrenal stress hormones are released during emotionally arousing tasks and enhance memory consolidation. NE cannot cross the blood-brain barrier. It binds to the vagal afferents of the nucleus of the solitary tract (=: NTS). These noradrenergic neurons project to the BLA, where NE is then released (they also project indirectly to the BLA via the locus coeruleus).\\
	Glucocorticoids freely enter the brain and bind to GR of noradenergic neurons in the brainstem that project to the BLA.
	% not sure if this subsection could have been written better  
	
	\section{PART 3 - Animal Models of human affective disorders}
	In the human population, the continuous bimodal model proves useful. After a certain threshold, an individual is said to have reached the state of a psychiatric disorder or has gained a state marker. A \textbf{state marker} is a noticeable change of a wild type phenotype, such as altered processing of aversive/rewarding life events or the attachment of emotional value to a stimuli (similar to a biomarker, but on a cognitive level most of the time - it can also relate to the activity of a brain area). Note: A biomarker is also a state marker.
	
	\subsection{Changes in the neurocircuitry in depression}
	In depression, the output of the amygdala is too high (also a \textbf{state marker}), while the PFC and the ACC are impaired amongst other brain areas. Normally, the ACC reacts to aversive stimuli and inhibits thus the amygdala (which also reacts to aversive stimuli). In depression, this connection is disturbed. Therefore, the output of the amygdala to the subgenual cingulate is too high. Due to this overreactivity, depressed people react stronger to aversive stimuli and experience negative emotions stronger while reward-related emotions are compromised. Furthermore, there are changes in the serotonin and dopamine system (decrease in serotonin and dopamine in general, which is related to the VTA and NAcc).\\
	Helplessness and uncontrollability are non-genetic factors for depression. Situations of helplessness or uncontrollability increase the activity of dACC (communication disturbed in depresssion $\Rightarrow$ dACC does not inhibit amygdala, therefore enhanced feeling of helplessness, while in controls, dACC would regulate amygdala output).
	
	\subsection{Optogenetics}
	Express an ion channel in a neural subpopulation by either injecting an adenovirus with the necessary genes on it or by producing a CRE-loxP animal. A CRE-loxP mouse needs one parent with the Cre gene in its genome under a promoter that occurs in a neural subpopulation of the brain and the other parent with the loxP sites flanking the gene of interest. Mating these two will result into offsprings with loss of gene genotype.\\
	Channelrhodopsin 2 depolarizes the cell upon light activation: The neurons are activated and it can thus reverse phenotypes such as decreased interest in 2-bottle test due to CSD or CUMS when it is expressed in the VTA (you can for example use the promoter for dopamine decarboxylase, since the VTA produces dopamine uniquely).\\
	Halorhodopsin hyperpolarizes neurons and thus acts inhibitory on neural firing upon light activation. It can reduce interest in reward when it is expressed in the VTA. (Use the adenovirus approach to create such mice if there is no extremely specific promoter which is uniquely active in that neural subpopulation).\\
	(Dopamine decarboxylase promoter for VTA. Promoter of enzyme for serotonin synthesis in dorsal Raph\'e nucleus. A myelin promoter for oligodendrocytes. This also allows temporal control if certain promoters are only active during a specific stage in development.)
	
	
	
	\textbf{General pathway:} senses $\rightarrow$ thalamus $\rightarrow$ amygdala and ACC, amygdala $\rightarrow$ subgenual cingulate and dlPFC, cortex $\rightarrow$ ACC $\rightarrow$ inhibits amygdala.\newline
	
	Psychiatric genetic approaches include endophenotypes, gene-environment interactions and relationship between gene and behaviour (SNPs for example correlate with psychiatric disorders). 
	
	\subsection{Endophenotypes: CRFR1 and 5-HTTLPR}
	\textbf{Def. Endophenotype: }Often, a complex disease cannot be studied directly. A complex disease consists of various aspects regarding behavioural, biochemical, neuronal, genetic etc. components. These can be studied as a substitute for the disease. If such a phenotype is linked to or part of a complex disease it is called an \textbf{endophenotype} ($\Leftrightarrow$ clinical phenotype $\Leftrightarrow$ risk allele if it has a genetic basis).
	\paragraph{Example of genotype-environment interaction: 5-HTTLPR.} There are two variants of this allele: long and short. Those with the short allele, have less efficient 5-HTT, while those with the long allele have more 5-HTT. The number of stressful life events correlates with the s 5-HTTLPR. These people have a higher prevalence of developing a depression throughout their lifetime. Furthermore, s 5-HTTLPR carriers experience fearful stimuli stronger than l 5-HTTLPR carriers, even if they are completely healthy. $\Rightarrow$ s 5-HTTLPR people have a more active amygdala when exposed to fearful stimuli $\Rightarrow$ 5-HTTLPR is an endophenotype!\newline
	
	An increase in CRFR expression by supplying multiple copies of the gene in the genome or with adenoviruses, the emotional reactivity of such an animal will be increased and its response to aversive stimuli also, since CRFR is also expressed on the amygdala and pituitary gland $\Rightarrow$ increased emotional learning and reactivity
	
	\subsection{Types of tests}
	An animal model must be able to validate certain aspects of a disease that are relevant for humans:\\
	\textbf{Aetiological validity: }All humans/animals with the same disease have the same underlying causes. Ideally, one exposes the animal to the same environmental factors to reproduce the disease. \\
	\textbf{Face validity: }The animal model recapitulates (parallels) important aspects of a complex human disease with regard to the behavioural, anatomical, biochemical, neuronal, cellular etc. components of a complex human disease. For example, a mouse shows the same behaviour towards a stimulus like a human. \\
	\textbf{Predictive validity: }A treatment applied to an animal model will behave similarly in humans. Such a model normally includes a drug that can reverse the changes. Also, if a drug takes weeks to become noticeable it can serve as an indicator in the human case. \\
	\textbf{Construct validity: }The methods used with which an animal model has been developed. Methods can be: CRISPR/Cas9 (knock-in or knock-out), viral infection, transgenic methods. \\
	In rodents, several tests have been established to capture specific aspects of a disease.\newline
	
	\textbf{Early-life stress} is an aetiological factor for depression (a test or model that has early-life stress such as separation of baby from mother (does not work in rodents, because it is normal, but it can be used in monkeys) has aetiological validity). Such individuals have a higher probability of developing depression. In animal experiments, such animals experience rewarding stimuli less pleasant and are easier subject to helplessness and enhanced fear learning (overvaluation of negative stimuli by the amygdala). Lastly, such a mouse will lose interest in reinforcement tests, where a mouse has to do progressively more to obtain a reward.
	
	\subsubsection{Chronic Social Defeat (CSD)}
	\textbf{Symptoms: }Helplessness, anhedonia, social withdrawal.\\
	Two mice are separated by a transparent wall. One mouse has an aggressive character and will beat up and bite the mouse when the wall is lifted. This is done for 1-10 mins over a course of weeks. The shy mouse will develop depressive symptoms such as social withdrawal, reduced interest in reward, increased reactivity towards aversive stimuli (results into \textbf{hyper-fear conditioning}) and reduced cognitive flexibility.\\
	CSD has \textbf{face validity}.\\
	\textbf{Additional information on CSD:} CSD results into generalized helplessness: Mice have a higher latency in escape behaviour and failure rate is higher $\Rightarrow$ no escape leads to the stimulus being experienced as uncontrollable. On a molecular level, oligodendrocytes (myeling) are compromised and expression of GABA interneurons and GPCRs is reduced in the amygdala amongst other brain areas.\\
	There is an increase in pro-inflammatory markers such as TNF-$\alpha$ and IFN-$\gamma$ and enhanced kynurenine pathway activity.
	
	\subsubsection{Chronic Unpredictibale Mild Stress (CUMS)}
	\textbf{Symptoms: } Biobehavioural changes due to CUMS.\\
	The mouse is exposed to arbitrary, stressful environmental changes throughout the day and night. Thus, it cannot adapt to an underlying structure of changes and will perceive the environment as unpredictable. Such mice show a decreased interest in sucrose preference in the 2-bottle test. (WRITE MORE)\\
	CUMS has \textbf{aetiological and face validity}.\\
	\textbf{Additional information on CUMS:} In combination with the 2-bottle test and a dopamine agonist or antidepressants (antidepressants take 2-3 weeks to show an effect), the effects of CUMS can be reversed (\textbf{aetiological, face and predictive validity}). Furthermore, CUMS leads to anhedonia: Self-stimulation in CUMS rats was decreased $\Leftrightarrow$ these rats needed more Amp\`ere compared to control rats during self-stimulation $\Rightarrow$ VTA is sensitive to CUMS. This phenotype can be reversed when stopping CUMS. Also, CUMS leads to an increase in floating during the forced swim test (\textbf{aetiological validity}, but interpretation is not straightforward).
	
	\subsubsection{2-way Escape test}
	\textbf{Symptoms: } Learned helplessness.\\
	2-way escape test simulates uncontrollable environments. Mice are placed in a box, where they will receive electric footshocks. The control group has the possibility to escape out of this environment. The uncontrollable group does not have this possibility. They show less escape behaviour (since trying would be in vain). $\Rightarrow$ they learned helplessness! 
	It has \textbf{face validity} (it is not a model, only a readout test).
	
	\subsubsection{Forced Swim Test}
	\textbf{Symptoms: } None (increase stress).\\
	Only a \textbf{readout} test! A mouse is put in water. It will struggle and try to get out of the water. A depressed mouse shows a decrease in struggle and an increase in time spent floating. This is sometimes taken as an indicators for depressive symptoms.
	It has \textbf{aetiological and predictive validity} \textbf{check if this is really correct!}.\\
	\textbf{Critique: }It is not a direct model or test for depression. Floating behaviour not completely clear and open to interpretation: Is this passive coping, energy saving or did the mouse give up? Such a test has to be meaningfully combined with other manipulations to infer meaningful results.
	
	\subsubsection{2-Bottle test (sucrose preference test)}
	\textbf{Symptoms: }Anhedonia, decrease in reward-wanting.\\
	Mice have the choice between two different bottles filled with either water (neutral) or sucrose (pleasant). Initially, both groups will choose the sucrose bottle with the same probability. After some time, the control group will choose sucrose more frequently than in the beginning, while the probability of choosing sucrose in depressed mice will decrease. 50\% means that the mouse does not care whether it drinks water or sucrose.\\
	In combination (or alone) with CUMS, it has \textbf{aetiological, face and predictive validity}.
	
	\subsubsection{Water maze}
	The animal is placed in water with only a small platform above water. On the sides of the bucket, there are landmarks such that it can orient itself. One can measure its trajectory to see the way the mouse swam. This test is used to test spatial memory.
	
	
	\subsection{Depression and the Immune System}
	\textbf{Pathway:} Amygdala $\rightarrow$ medulla $\rightarrow$ SNS $\Rightarrow$ amygdala activation of SNS increases ACTH at pre-ganglionic side which results into release of NE at the post-ganglionic side $\Rightarrow$ NE binds to NE receptors of macrophages $\Rightarrow$ release of TNF (pro-inflammatory molecules) and general increase inf NF-$\kappa$B signalling. Also, there is a cortisol increase. In a test setting, this can be simulated by the TSST in humans. Also, HPA axis is activated during stressful conditions. NE receptor antagonists\footnote{In the brain, propranolol is able to pass the blood-brain barrier and reduce IL-1$\beta$ levels there} can reduce IL-1$\beta$ levels (SNS signalling blocked).\newline
	
	Cytokines are the messengers of the immune system capable of passing the BBB. In the CNS, cytokines reduce monoamine activity (5-HT, DA, NE), reduce neurogenesis in the hippocampus, increase toxic glutamate signalling, reduce mitochondrial function due to oxidative stress and dysregulate astrocyte-neuron interactions.\\
	In humans, SNPs related to inflammation as well as increased pro-inflammatory cytokines were observed $\Rightarrow$ evidence for aetio-pathophysiology of depression. During some types of inflammation, the subgenual ACC is hyperactive resulting into increased processing of negative stimuli. The endotoxin LPS can be applied to induce depression-relevant behaviour and biological phenotypes (LPS is stressful) $\Rightarrow$ decreased preference in the 2-bottle test.\\
	Lastly, pro-inflammatory cytokines such as IFN-$\gamma$ or TNF-$\alpha$ increase IDO activity $\Rightarrow$ kynurenine pathway preferred over serotonin pathway: tryptophan $\rightarrow$ kynurenine $\rightarrow$ 3-Hydroxykynurenine $\rightarrow$ quinolic acid $\rightarrow$ nicotinamide. 3-Hydroxykynurenine is toxic for the brain and quinolic acid agonizes on NMDAR, leading to increased NMDAR signalling which enhances depressive symptoms. Kynurenine can be converted into kynurenic acid which inhibits NMDAR through exercising. In the skeletal muscles, the enzyme KAT is produced which catalizes the conversion.
	
	
	\section{PART 4 - Pre-clinical Psychopharmacology}
	\subsection{The serotonin system}
	% add occurrence of HT1A/2A/2C in the brain. 
	Serotonin is exclusively produced in the (medial and dorsal) Raph\'e nucleus. It projects to other brain areas such as the hippocampus, amygdala, ACC, VTA and the NAcc. Serotonin is produced out of tryptophan via 5-Hydroxy-L-tryptophan to 5-Hydroxytryptamine (5-HT, also known as serotonin). \\
	Serotonin receptors occur in different brain areas. There are 13 of them in total. For depression, HT$_{1A}$, HT$_{2A}$ and HT$_{2C}$ are important. Serotonin binds to all of these at the post-synaptic side. There are also 5-HTTs that import synaptic serotonin back into the pre-synapse. In the pre-synapse back again, serotonin is catalysed into an inactive compound via MAO-A (monoamine oxidase A), which is increased in depressed patients. Thus, serotonin is present in fewer quantities in the synaptic cleft and cannot as effectively modulate other brain areas. Also, HT$_{1A}$ is located at the pre-synaptic side. Drugs that act on the HTT are called selective serotonin reuptake inhibitors (SSRIs). Inhibition of 5-HTT leads to elevation in mood.
	\paragraph{}\textbf{Serotonin synthesis intact: }In depressed patients, there is an imbalance regarding the sensitivity of the receptors. HT$_{1A}$ becomes insensitive, while HT$_{2A}$ and HT$_{2C}$ are overly sensitive. HT$_{1A}$ is colloquially called the "good" receptor, since a decline in activity leads to depressed mood. HT$_{2C}$ over-sensitivity increases depressed mood sensation. Thus, too much serotonin arrives at the HT$_{2C}$ receptor leading to a decline in well-being, while not enough is reached at the HT$_{1A}$. SSRIs such as Fluoxetine inhibit HTT at the pre-synaptic side, such that more serotonin is available in the synaptic cleft and can thus reach more HT$_{1A}$ increasing its sensitivity to normal while decreasing the over-sensitivity of HT$_{2C}$ to normal.
	\paragraph{}\textbf{Serotonin synthesis disturbed - Inflammation hypothesis: }
	In inflammation-related depressed patients, not enough serotonin is being produced. Stress leads to a decrease of immune functions. Inflammation leads to an increase of the conversion of tryptophan to kynurenine via IDO, also known as the \textbf{kynurenine pathway}, which is mediated by cytokines. Thus, not enough tryptophan is being produced into serotonin and one has a lack thereof. IDO inhibition has been proposed as a therapeutic target. (\textbf{Side note: }Temporary stress increases serotonin availability in the synaptic cleft, while there is only normal reuptake. This has been linked to an increase in emotional reactions)
	\subsection{5-HTT gene-linked polymorphic region (5-HTTLPR)}
	5-HTTLPR occurs in a short or long form. The short 5-HTTLPR results into a decrease of 5-HTT. Thus, reuptake is less effective as in the long form. This is also known as a natural antidepressant. At the same time, it is a major genetic risk factor for developing a depression, since 5-HTT function/expression is compromised. This endophenotype generally leads to an increased reactivity to aversive stimuli. Childhood stress increases the chance of developing a depression in adult age with the short 5-HTTLPR.
	\subsection{Next Generation Antidepressants: Agomelatine}
	Agomelatine is a new type of antidepressant acting on both melatonin receptors M1 and M2 as an agonist in the SCN as well as inhibiting HT$_{2C}$ as an antagonist (the SCN has 5-HT$_{2C}$ and melatonin receptors\footnote{The SCN has a constant baseline signalling as long as there is light. Light is being perceived by special receptors in the eye that signal to the SCN, thus ensuring its activity. Through GABA signalling, the SCN can inhibit the pineal gland. Darkness inhibits the signalling of the SCN, thus SCN does not inhibit the pineal gland, which, amongst other things, secretes melatonin.}). Thus, agomelatine can treat both sleep disorders and depression. Furthermore, it results into an increase of DA and NA.\\
	\textbf{Mechanism of agomelatine: }Agomelatine binds to HT$_{2C}$ of GABAergic interneurons. These would normally inhibit other dopaminergic neurons. Agomelatine inhibits these interneurons, thus DA signalling is increased in the dopaminergic neurons.
	\paragraph{} In the SCN, agomelatine binds to the M1/2 receptors. There, it is a more potent regulator for circadian rhythm and sleep induction than melatonin, since its binding to the receptors is more potent than melatonin $\Rightarrow$ agomelatine resets the SCN (it inhibits the firing of these cells).\\
	\textbf{Benefits of agomelatine: }improved sleep, preservation of sexual function, antidepressant properties, synchronzation of circadian rhythm, anxiolytic properties; reversal of CUMS in rats (recovery of sucrose consumption)
	%benefits of agomelatine as a list
	
	\paragraph{}\textbf{Ketamine as a potential treatment: }Ketamine is a psychedelic drug that acts on the glutamate system as a non-competitive inhibitor of NMDA receptors (antagonist). It has been proposed that it activates neurotrophins and synaptogenesis. It inhibits the entry of Ca$^{2+}$ into the cell and Na$^{+}$ to a lesser extent. After glutamate release of the pre-synaptic cell and ketamine administration, ketamine binds to NMDAR, such that glutamate can only bind to AMPAR. This results into an increase of AMPAR signalling at the post-synaptic side inducing enhanced synthesis and trafficking of AMPARs and release of neurotrophic factors. This leads to enhanced synaptogenesis (binding of neurotrophins, such as BDNF to survival receptors such as Trk) and neuroplasticity.\\
	Ketamine has been shown to effectively counteract inflammation-related depression: Quinolic acid is an intermediate molecule in the kynurenine pathway which acts as an NMDAR agonist, thus enhancing NMDAR signalling at the post-synaptic side and worsening depressive symptoms. Ketamine inhibits the enzyme IDO, such that tryptophan cannot enter the kynurenine pathway (therefore, it is likely to be converted into serotonin again instead of kynurenine). Furthermore, administration of LPS (endotoxin) leads to an increase of cytokines and thus, to the activation of the kynurenine pathway. Ketamine reverses these effects. Therefore, ketamine is protective against LPS.
	
	%\newpage
	\section{Model answers and questions}
	% the formatting is a bit off here.
	
	\subsection{Set 1}
	\textbf{Question 1:} Describe a mechanism of antidepressant treatment other than selective serotonin reuptake inhibition. You can choose either another approved antidepressant or a drug for which there is some evidence that it has an antidepressant effect.\newline
	
	Agomelatine: Agomelatine acts on both M1/M2 receptors for melatonin in the suprachiasmatic nucleus (SCN) as well as on HT$_{2C}$ receptors. It treats sleep disorders and depression at the same time. Agomelatine is a more potent agonist than melatonin and thus, it resets the circadian rhythms more effectively and induces sleep faster. Furthermore, it is a good antagonist on the HT$_{2C}$ receptors. Therefore, it inhibits the HT$_{2C}$ receptor based signalling by blocking its binding site. This results into serotonin being unable to bind on the HT$_{2C}$ receptors and therefore, more serotonin can bind to the HT$_{1A}$ receptor and activate its signalling cascade. The HT$_{2C}$ receptor has been linked to decreased mood and motivation and generally to depressive symptoms, while HT$_{1A}$ is colloquially called the "good serotonin receptor", since its signalling results into enhanced mood.\\
	Agomelatine binds to the HT$_{2C}$ receptor of GABAergic neurons in the SCN, which normally inhibit dopaminergic neurons. Since it is as antagonist, dopaminergic neurons are not inhibited by GABAergic interneurons anymore and thus, there is an increase in dopamine signalling.\\
	Treatment with agomelatine does not affect sexual function, while SSRIs might result into a decline in sexual activity.\newline
	
	\textbf{Question 2:} Combine a manipulation (e.g. genetic, environmental) and a behavioural test to produce an animal model of depression. Which brain structures underlie the behavioural changes that you observe in your model?\newline
	Buy 5-HTTLPR knockout mice and expose them early on to chronic unpredictable mild stress. Due to their genotype, they are highly likely to develop symptoms of depression such as anhedonia (decrease in reward wanting). Test the behavioural phenotype with the 2-bottle test. The control group will tend to drink more from the sucrose bottle after a while, while the depressed mice will have a decrease in the likelihood of choosing the sucrose bottle (50$\%$ means no preference regarding sucrose or water). The underlying brain structures are the VTA, the NAcc and VP. We can observe their change in dopamine signalling, especially the change in phasic signalling.\newline
	
	
	\textbf{Question 3:} Describe the functions of dopamine neurotransmission in reward processing. Discuss whether changes in dopamine function might be relevant to depression.\newline
	
	During reward processing, dopamine signalling will undergo certain changes. There is always a baseline firing of dopamine neurons in the VTA and NAcc. Moreover, these neurons are also inhibited by GABAergic interneurons, but in total, dopamine signalling is more potent and we have a positive value of DA signalling. When exposed to a likeable reward, there will be a phasic increase in dopamine signalling (short bursts of DA signalling in VTA and NAcc neurons = high increase). Since NAcc DA neurons also project to cortical areas such as the PFC, this increase in DA signalling will result into trying to figure out the circumstances that lead to the consumption of the reward. After repeated exposure to the same reward, DA signalling will remain at baseline, since the sensation of the reward is already known and there is no reward prediction error (RPE). When the reward is removed from the experimental setting, there will actually be a decrease in DA signalling at the time the reward was expected to be delivered. Thus, we have a negative RPE. Dopamine is central in reward-seeking behaviour, motivation and also learning.\\
	Changes in dopamine function normally result into a decrease in mood and motivation. These changes are highly relevant for depression, as they result into anhedonia for example, which is a characteristic of depression.\newline
	
	\subsection{Set 2}	
	\textbf{Question 1:} Describe the criteria that are used to assess the “validity” of an animal model for a human psychiatric disorder?\newline
	4 criteria: aetiological, face, predictive and construct validity: \\
	Aetiological and Face validity: All humans/animals with the same disease have the same underlying factors contributing to the disease or endophenotype. For model organisms, the animals must have important or the same aspects of a human disease with the human underlying factors, such as genetic mutations, changes in cell morphology or connectivity, biochemical changes, behavioural changes.\\
	Predictive validity: An animal inflicted with a disease can be treated with a drug or genetic/environmental manipulation such that it reverts to the original (healthy) phenotype. The applied drug/manipulations are predictive for the human case. Especially, if a drug takes week to become noticeable in the animal, this might also be similar in the human case.\\
	Construct validity: The validity under which conditions a model is created. This includes also genetic manipulations such as transgenic manipulations and CRISPR/Cas9.\newline
	
	
	\textbf{Question 2:} Describe the effects of the hormone cortisol/corticosterone on learning and memory.\newline
	Cortisol occurs in humans, while corticosterone occurs in mice. They are functionally identical, since both bind to GR and MR. MR only occurs in the BLA and CNA while GR occurs globally in the brain. MR has a 10 times higher affinity for cortisol than GR. Cortisol, since it is a steroid, can freely pass the blood-brain barrier and bind to GR of noradrenergic cells in the brain stem, which project to the amygdala. Cortisol enhances the speed of learning of emotional input. For example, mice overexpressing the MR will quicker undergo fear conditioning than controls. In general, emotional information learning and consolidation is accelerated. In the case of fear learning, MR is needed for fear conditioning in the amygdala, but not for consolidation. MR KO has shown that the animal does not retain fear memories in footshock conditioning. They will not show any freezing behaviour. GR is needed for the consolidation, but not for the initial conditioning (learning) stage of fear learning. GR KO has shown that initially, mice will undergo a fear response in footshock experiments (freezing), but when the same experiment is repeated 24 hours later, the animal will not show freezing behaviour initially (as if it has forgotten the chamber was dangerous).\\
	Persistently increased levels of cortisol have a negative influence on memory retrieval. Cortisol is only released when under stress which can be induced through LPS (endotoxin - will also induce an inflammation response) or when exposed to psychological stress. During short time periods, cortisol mainly binds to MR and GR will only be bound to when stress response occurs at longer time scales. the vmPFC also has GR and when cortisol binds to it, working memory and memory retrieval is impaired (temporarily, can be restored when cortisol levels decrease).\newline 
	
	\textbf{Question 3:} Describe and compare classical/Pavlovian conditioning and operant/instrumental conditioning. Describe an experimental paradigm where these two learning processes are combined in order to study the neurobiology of motivation.\newline
	Pavlovian conditioning: Does not require the animal to do action. It is independent of belief, desire and instrumentality criteria. Co-occurrence is neither sufficient nor necessary for classical conditioning to occur, but helpful to have the stimuli in relatively quick succession to make the expected associations. (Diagram). CS (tone) is paired with a US (smell of food) which leads to a CR and UR. CR = UR must not always be the case. In the former, the CR=UR=salivate. In mice fear conditioning, the UR = running wildly when exposed to footshock (US) and CR = freezing when CS = tone is presented.\\
	Operant conditioning requires more acivity from an animal, such as pressing a lever. It is dependent on the belief, desire and insturmentality criteria: Belief: if I perform this action, I will get this reward. Desire: I want this reward (especially, this reward has motivational and affective value). A reward can also be "not getting a footshock" or so. Initially, the operant stimulus (OS) is a neutral stimulus, such as the presence of a lever. Through random exploration from the animal's side, it will press the lever and it will be presented with food (reward). This reinforces the OS, such that the OS is intentional $\Leftarrow$ the OS becomes the stimulus (S) and has become associated with a positive affect. Then, S $\leftarrow$ R (reward = outcome (O)).\\
	One can combine these two settings such that a pavlovian CS leads to the outcome (reward) of the operant conditioning (OR-theory) case. The CR and outcome from operant conditioning need not necessarily be the same. The association will occur anyway. Thus, one can study motivation in animals (it is also helpful to have electrodes implanted in the VTA and NAcc to measure the spiking of the dopamine neurons there): First do classical conditioning with CS = tone, separately. Then, perform OP conditioning with OS = lever and reward = food $\Leftarrow$ food becomes outcome. The food fulfills all three criteria from above. Then, place a mouse in the chamber and play the tone, such that it starts searching for the lever to obtain he reward (= outcome). Moreover, the VTA and NAcc will show short phasic dopamine pulses (short increased spikes) when undergoing each step to get a reward. One can study this more accurately by exposing the mice to two levers (= OS1 and OS2) before they get their reward. This will show an increase each time it has to perform an action prior to the actual performance of the lever press.\\
	Write on RPE: leaving out here, but in exam, do it anyway. Better write more, than less.\newline
	
	\subsection{Set 3}
	\textbf{Question 1:} Describe fear conditioning in terms of behavior and neurobiology.\newline
	Behaviour: The animal adapts a new type of behaviour for a given stimuli, which in evolutionary terms should maximize its survival. A mouse in a chamber undergoes classical conditioning (diagram here) with CS = tone or light and US = footshock. UR = running in a frenzy, CR will become eventually a freezing behaviour upon sensing the CS. The neurobiological basis is to a great extent the amygdala and its three main divisions the lateral amygdala (LA) the baso-lateral amygdala (BLA) and the central nucleus of the amygdala (CNA).\\
	Functions of the three main subdivisions (diagram here): LA receives sensory inputs from the thalamus and relays them to the BLA where the main processing occurs. The processed output is relayed to the CNA with gives the final output. It has been shown that the LA is necessary for fear conditioning. Lesion studies have shown that animals with a LA will not undergo fear conditioning and therefore, they will never show freezing behaviour in a footshock chamber. The CNA is necessary for the consolidation of fear learning finally. Lesion studies have shown that animals lacking a CNA will not show freezing behaviour 24 hours later. Furthermore, oxytocing as well as CRF antagonists can be applied to the CNA and block fear learning.\\
	On a molecular level, the LA undergoes LTP-like changes. Fear conditioning increases AMPAR counts on the postsynaptic side in the LA.\newline
	
	\textbf{Question 2:} Describe the effects of the hormone cortisol/corticosterone on learning and memory.\newline
	See above.\newline
	
	\textbf{Question 3:} For one symptom of depression, describe an animal model that could be used to study this symptom. Which brain regions and neurotransmitters regulate this symptom?
	SOLVE THIS PART AGAIN \newline
	
	\subsection{Set 4}
	\textbf{Question 1:} Describe the effects of the hormone cortisol/corticosterone on learning and memory.\newline
	See above.\newline
	
	\textbf{Question 2:} Describe and compare classical/Pavlovian conditioning and operant/instrumental conditioning.\newline
	See above.\newline
	
	\textbf{Question 3:} Describe the Neuroanatomy and function of the amygdala. Which changes can be observed in patients with depression?
	Neuroanatomy: The amygdala is made up of three main subdivisions, namely the lateral and basolateral amygdala and the central nucleus of the amygdala. It occurs on both hemisphere and it is posterior to the hippocampus and it is almond-shaped (looks like a Mandelkern).\\
	Functions of the amygdala subdivisions: LA: receives sensory input from the thalamus and it is involved in fear conditioning. It undergoes LTP-like changes in fear conditioning (see above).\\
	BLA: The BLA performs the main computation of the stimuli, which it receives from the LA and relays them to the CNA. Needed in fear conditioning.\\
	CNA: Involved in fear conditioning; it is the output area to other brain structures.\\
	(Diagram of amygdala-related neurocircuitry in depression): dACC normally inhibits the amygdala such that the output is not so extreme. Moreover, during uncontrollable environments, the dACC is even more active (= more spiking) such that it inhibits the amygdala better. This inhibition is disturbed in depressed patients and they feel helpless faster and experience negative emotions stronger. dACC also gets its input from the thalamus. The input from thalamus to amygdala is increased in depressed patients and the output to the subgenual nucleus is also increased. (Talk more about the diagram).
	
	\section{Model questions and answers}
	Animal models: What are their problems to study complex human diseases (or something like that).
	
	
\end{document}