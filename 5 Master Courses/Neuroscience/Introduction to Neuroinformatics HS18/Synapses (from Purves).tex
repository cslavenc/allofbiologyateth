\documentclass[pt11, a4wide, twosided]{article}
\usepackage{a4wide}


\begin{document}
	\title{Synapses (from Purves)}
	
	\maketitle
	
	\section{Electrical synapses}
	The pre- and postsynaptic cell are connected via gap junctions which are made of \textbf{connexons}. A connexon is a hexamer of connexins. An electrical synapse consists of two connexons (one on each side) that are connected to one another (length$_{connexon}$ = 3.5 nm) forming a pore. This type allows for a very fast transmission without time delay as ions can freely pass through connexons as well as other molecules such as ATP. This type of pore allows for bidrectional ion flow depending from where the electrical signal originated.
	
	\section{Chemical synapses}
	Chemical synapses release neurotransmitters into the synaptic cleft for signal transduction. There is a delay until the post-synapse generates an PSP, since it is not virtually instantaneous as in electrical synapses.
	
	\paragraph{Molecular components of the exocytosis complex for neurotransmitter release.}The synaptic vesicles need to dock onto a molecular complex, such that they can successfully release the neurotransmitters into the synaptic cleft. This complex consists of synaptobrevin (a SNARE), syntaxin and SNAP-25. Synaptobrevin is on the outside of a vesicle and intercalates with syntaxin and SNAP-25 for docking. The vesicle is pulled to the membrane and Ca$^{2+}$ enters into presynaptic neuron. Ca$^{2+}$ binds to synaptotagmin, which is also located on the vesicle, which acts as a switch for neurotransmitter release.
	
	\paragraph{NT receptors.} There are two major types of NT receptors: ionotropic and metabotropic.\\
	Ionotropic receptors (also known as ligand-gated ion channels) open when the corresponding NT binds. Thus, ions flow in. Often, it is Na$^{+}$ for AMPARs and Ca$^{2+}$ for NMDARs and Na$^{+}$ to a lesser extent. Both are activated by glutamate. It is a fairly fast response: orders of ms.\\
	Metabotropic receptors are GPCRs. When its corresponding ligand binds to it, a G-protein coupled protein signalling cascade occurs. This results into a number of different metabolic steps with second messengers, such that in the end, at another location an ion channel will open up and allow ion flow. It is quite a slow response: ms to mins or even longer.
	
	
\end{document}