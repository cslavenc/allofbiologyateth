\documentclass[11pt, english]{article}

\usepackage{fontspec}			% specify fonts
    \defaultfontfeatures{Ligatures={TeX}} %gedankenstrich
	%\setmainfont{TeX Gyre Pagella}
	%\setmainfont{Linux Libertine O}	

% \usepackage[dvips,a4paper,top=2.5cm,left=2.5cm,right=2.5cm,bottom=2.5cm]{geometry} % control page layout
\usepackage{graphicx}			% embed graphics
\usepackage[export]{adjustbox}	% enable positioning of graphics
\usepackage{wrapfig}				% enable textwrapping around figures
\usepackage{hyperref}					% for better control of hyperlinks and \autoref{} command
	\hypersetup{colorlinks=false} 		% do not colour links
	\hypersetup{pdfborder={0 0 0}} 		% no boxes around links (if colorlinks=false)
\usepackage{babel}		% for German section headers, hyphenation etc. Must stay below hyperref so the section renaming works!
\usepackage{linguex}			% linguistic examples with glosses
	\let\eachwordone=\it		% first line of gloss in italics
\usepackage{mdwlist}			% nice lists
\usepackage{chronology}
\usepackage{amsmath}		%enhanced equations
\usepackage{amsfonts}	%more available symbols in equations
\usepackage{verbatim}
\usepackage{natbib}				% control bibliography
	\setcitestyle{aysep={},citesep={;},notesep={:}}		% set citation style: no separator between author and year, separator between multiple citations = semicolon, separator between citation and postnote (e.g. page number) = colon
\usepackage{scalefnt}			% arbitrary scaling for font size

\usepackage{abstract}
\renewcommand{\absnamepos}{flushleft}
\setlength{\absleftindent}{0pt}
\setlength{\absrightindent}{0pt}

\newcommand{\glosssize}{\scalefont{.85}}			% second line of gloss small
	\let\eachwordtwo=\glosssize						
\newcommand{\gender}{\textunderscore}			% separator for genderwashing
\newcommand{\source}[1]{\hfill (#1)\\[-0.2cm]}	% sources for examples
\newcommand{\sourcewrap}[1]{\\ \strut \hfill (#1)\\[-0.2cm]}	% sources for examples
\newcommand{\abrac}[1]{$\langle$#1$\rangle$}		% ⟨angle brackets⟩ with \abrac{}
\renewcommand{\labelitemi}{$\bullet$}			% change symbol used in lists
\renewcommand{\labelitemii}{$\circ$}				% change symbol used in lists

\pagenumbering{gobble}
\renewcommand{\section}[2]{}

\begin{document}

\begin{center}

%\vspace*{3cm} % for prescribed page margins

%	\includegraphics[height=17mm]{grafiken/uzh_logo_d_pos.pdf}\\[0.5cm] % this height is fixed by the university's corporate design rules
	\Large{%Universität Zürich\\
	\bfseries Readings in Neroinformatics} \\[1cm] 

%	{\huge \bfseries titel:\\
% untertitel}\\[3.5cm]

%	\Large{BA-Arbeit}\\[0.1cm]
%	\Large{Hauptfach (90 ECTS)} \\[0.1cm]
%	\Large{Vorgelegt im Juni 2015} \\[1.5cm]

	\Large{ Ephraim Seidenberg}\\[0.1cm]
	10-931-798\\[0.1cm]
%	\Large{ \bfseries Betreuung:}\\[5cm]
	
	\vfill

\end{center}

\vspace{1cm}

\bibliographystyle{unified-caps}
\bibliography{absbiblio}
\nocite{mead1990neuromorphic}

\begin{abstract}
\normalsize
Even the simplest animal brains are awesome computational instruments. Their elementary operations provide a rich set of powerful computational primitives. We can build devices that implement these primitives by the use of wafer-scale silicon fabrication, enabling us to build entire systems based on the organizing principles of the nervous system, many orders of magnitude more effective than what could be implemented up to the present time. To this end, we have to find a natural way to integrate computational primitives into an overall system-design strategy. Here I propose the term \emph{neuromorphic systems} to generically refer to such systems and give an outline of the prerequisites and important principles as well as the currently available approaches. I argue that adaptive techniques are needed to correct for differences between nominally identical components, and that this adaptive capability leads naturally to systems that learn about their environment. I show that the representation of information by the relative values of analog signals creates the principle advantage and I maintain that the basic two-dimensional limitation of silicon technology is not a serious limitation in exploiting the potential of neuromorphic systems. With these notions I show the next steps in this development and indicate where it may lead in the long run.
\end{abstract}



\noindent 206 words.
\end{document}