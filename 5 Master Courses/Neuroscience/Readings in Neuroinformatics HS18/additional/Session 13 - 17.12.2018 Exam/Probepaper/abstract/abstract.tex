\documentclass[11pt, english]{article}

\usepackage{fontspec}			% specify fonts
    \defaultfontfeatures{Ligatures={TeX}} %gedankenstrich
	%\setmainfont{TeX Gyre Pagella}
	%\setmainfont{Linux Libertine O}	

% \usepackage[dvips,a4paper,top=2.5cm,left=2.5cm,right=2.5cm,bottom=2.5cm]{geometry} % control page layout
\usepackage{graphicx}			% embed graphics
\usepackage[export]{adjustbox}	% enable positioning of graphics
\usepackage{wrapfig}				% enable textwrapping around figures
\usepackage{hyperref}					% for better control of hyperlinks and \autoref{} command
	\hypersetup{colorlinks=false} 		% do not colour links
	\hypersetup{pdfborder={0 0 0}} 		% no boxes around links (if colorlinks=false)
\usepackage{babel}		% for German section headers, hyphenation etc. Must stay below hyperref so the section renaming works!
\usepackage{linguex}			% linguistic examples with glosses
	\let\eachwordone=\it		% first line of gloss in italics
\usepackage{mdwlist}			% nice lists
\usepackage{chronology}
\usepackage{amsmath}		%enhanced equations
\usepackage{amsfonts}	%more available symbols in equations
\usepackage{verbatim}
\usepackage{natbib}				% control bibliography
	\setcitestyle{aysep={},citesep={;},notesep={:}}		% set citation style: no separator between author and year, separator between multiple citations = semicolon, separator between citation and postnote (e.g. page number) = colon
\usepackage{scalefnt}			% arbitrary scaling for font size

\usepackage{abstract}
\renewcommand{\absnamepos}{flushleft}
\setlength{\absleftindent}{0pt}
\setlength{\absrightindent}{0pt}

\newcommand{\glosssize}{\scalefont{.85}}			% second line of gloss small
	\let\eachwordtwo=\glosssize						
\newcommand{\gender}{\textunderscore}			% separator for genderwashing
\newcommand{\source}[1]{\hfill (#1)\\[-0.2cm]}	% sources for examples
\newcommand{\sourcewrap}[1]{\\ \strut \hfill (#1)\\[-0.2cm]}	% sources for examples
\newcommand{\abrac}[1]{$\langle$#1$\rangle$}		% ⟨angle brackets⟩ with \abrac{}
\renewcommand{\labelitemi}{$\bullet$}			% change symbol used in lists
\renewcommand{\labelitemii}{$\circ$}				% change symbol used in lists

\pagenumbering{gobble}
\renewcommand{\section}[2]{}

\begin{document}

\begin{center}

%\vspace*{3cm} % for prescribed page margins

%	\includegraphics[height=17mm]{grafiken/uzh_logo_d_pos.pdf}\\[0.5cm] % this height is fixed by the university's corporate design rules
	\Large{%Universität Zürich\\
	\bfseries Readings in Neuroinformatics} \\[1cm] 

%	{\huge \bfseries titel:\\
% untertitel}\\[3.5cm]

%	\Large{BA-Arbeit}\\[0.1cm]
%	\Large{Hauptfach (90 ECTS)} \\[0.1cm]
%	\Large{Vorgelegt im Juni 2015} \\[1.5cm]

	\Large{ Ephraim Seidenberg}\\[0.1cm]
%	10-931-798\\[0.1cm]
%	\Large{ \bfseries Betreuung:}\\[5cm]
	
	\vfill

\end{center}

\vspace{1cm}

\bibliographystyle{plainnat}
\bibliography{absbiblio}
\nocite{cook2008reusable}

\begin{abstract}
\normalsize
Neural networks implement rules according to which they can process input and generate output. This can be achieved through training with logical data and has been enhanced with probabilistic capabilities. Because however these rules have to be followed throughout the neural network, they have to be learned at every part. Repeating this learning process over and over again leads to a large redundancy, growing together with the size of the neural network itself. In this paper I introduce the \textit{reusable symbol problem} and I propose that its solution would remove much of that redundancy. I present a concrete canonical instance integrating the Wang tiling problem. The Wang tiling problem structures the neural network graphically into a workspace and a clearly scaled set of allowed tiles which can be arranged according to the color of their edges. I argue that reaching this milestone by applying reusable symbols to solve the Wang tiling problem can be regarded as a first milestone to solve the reusable symbol problem in general. Setting this milestone is intended to encourage work aimed at a better understanding of the symbolic processing performed by neural systems.
\end{abstract}


\noindent 188 words.
\end{document}