\documentclass[11pt, english]{article}

\usepackage{fontspec}			% specify fonts
    \defaultfontfeatures{Ligatures={TeX}} %gedankenstrich
	%\setmainfont{TeX Gyre Pagella}
	%\setmainfont{Linux Libertine O}	

% \usepackage[dvips,a4paper,top=2.5cm,left=2.5cm,right=2.5cm,bottom=2.5cm]{geometry} % control page layout
\usepackage{graphicx}			% embed graphics
\usepackage[export]{adjustbox}	% enable positioning of graphics
\usepackage{wrapfig}				% enable textwrapping around figures
\usepackage{hyperref}					% for better control of hyperlinks and \autoref{} command
	\hypersetup{colorlinks=false} 		% do not colour links
	\hypersetup{pdfborder={0 0 0}} 		% no boxes around links (if colorlinks=false)
\usepackage{babel}		% for German section headers, hyphenation etc. Must stay below hyperref so the section renaming works!
\usepackage{linguex}			% linguistic examples with glosses
	\let\eachwordone=\it		% first line of gloss in italics
\usepackage{mdwlist}			% nice lists
\usepackage{chronology}
\usepackage{amsmath}		%enhanced equations
\usepackage{amsfonts}	%more available symbols in equations
\usepackage{verbatim}
\usepackage{natbib}				% control bibliography
	\setcitestyle{aysep={},citesep={;},notesep={:}}		% set citation style: no separator between author and year, separator between multiple citations = semicolon, separator between citation and postnote (e.g. page number) = colon
\usepackage{scalefnt}			% arbitrary scaling for font size

\usepackage{abstract}
\renewcommand{\absnamepos}{flushleft}
\setlength{\absleftindent}{0pt}
\setlength{\absrightindent}{0pt}

\newcommand{\glosssize}{\scalefont{.85}}			% second line of gloss small
	\let\eachwordtwo=\glosssize						
\newcommand{\gender}{\textunderscore}			% separator for genderwashing
\newcommand{\source}[1]{\hfill (#1)\\[-0.2cm]}	% sources for examples
\newcommand{\sourcewrap}[1]{\\ \strut \hfill (#1)\\[-0.2cm]}	% sources for examples
\newcommand{\abrac}[1]{$\langle$#1$\rangle$}		% ⟨angle brackets⟩ with \abrac{}
\renewcommand{\labelitemi}{$\bullet$}			% change symbol used in lists
\renewcommand{\labelitemii}{$\circ$}				% change symbol used in lists

\pagenumbering{gobble}
\renewcommand{\section}[2]{}

\begin{document}

\begin{center}

%\vspace*{3cm} % for prescribed page margins

%	\includegraphics[height=17mm]{grafiken/uzh_logo_d_pos.pdf}\\[0.5cm] % this height is fixed by the university's corporate design rules
	\Large{%Universität Zürich\\
	\bfseries Readings in Neuroinformatics} \\[1cm] 

%	{\huge \bfseries titel:\\
% untertitel}\\[3.5cm]

%	\Large{BA-Arbeit}\\[0.1cm]
%	\Large{Hauptfach (90 ECTS)} \\[0.1cm]
%	\Large{Vorgelegt im Juni 2015} \\[1.5cm]

	\Large{ Ephraim Seidenberg}\\[0.1cm]
%	10-931-798\\[0.1cm]
%	\Large{ \bfseries Betreuung:}\\[5cm]
	
	\vfill

\end{center}

\vspace{1cm}

\bibliographystyle{unified-caps}
\bibliography{absbiblio}
\nocite{hubel1968receptive}

\begin{abstract}
\normalsize
To process visual information, there needs to be a mechanism integrating separate light stimuli into one visual experience. After having studied the involved functions in the cat visual pathway, the question remains as to how these insights can be extended to the human visual system. With visual capabilities very similar to human's especially in acuity and color vision, we chose to perform new experiments in the monkey striate cortex where we introduced electrodes at various cortical sites under light anesthesia. With seven penetrations in six spider monkeys, we performed a series of recordings from 150 cells, and another from 272 cells with 25 penetrations in 16 macaque monkeys. From the response patterns to various light stimuli, we were able to distinguish between three cell types and their prevalence at different cortical layers; \textit{simple cells}, sensitive to light stimuli with a particular `on' and `off' areas, \textit{complex cells}, most sensitive to particular orientations of a light bar on any part of their receptive field, as well as \textit{hypercomplex cells}, sensitive to both the particular shape and orientation of a light bar. We also found that most cells were binocularly driven and most weren't color sensitive. These findings suggest that in the striate cortex the input from the two eyes converges into single cells and increasingly complex fields are elaborated from geniculate concentric fields. Now, further examinations about later stages of information integration can be made for a more complete model of the human visual pathway.
\end{abstract}



\noindent 249 words.
\end{document}