\documentclass[11pt, english]{article}

\usepackage{fontspec}			% specify fonts
    \defaultfontfeatures{Ligatures={TeX}} %gedankenstrich
	%\setmainfont{TeX Gyre Pagella}
	%\setmainfont{Linux Libertine O}	

% \usepackage[dvips,a4paper,top=2.5cm,left=2.5cm,right=2.5cm,bottom=2.5cm]{geometry} % control page layout
\usepackage{graphicx}			% embed graphics
\usepackage[export]{adjustbox}	% enable positioning of graphics
\usepackage{wrapfig}				% enable textwrapping around figures
\usepackage{hyperref}					% for better control of hyperlinks and \autoref{} command
	\hypersetup{colorlinks=false} 		% do not colour links
	\hypersetup{pdfborder={0 0 0}} 		% no boxes around links (if colorlinks=false)
\usepackage{babel}		% for German section headers, hyphenation etc. Must stay below hyperref so the section renaming works!
\usepackage{linguex}			% linguistic examples with glosses
	\let\eachwordone=\it		% first line of gloss in italics
\usepackage{mdwlist}			% nice lists
\usepackage{chronology}
\usepackage{amsmath}		%enhanced equations
\usepackage{amsfonts}	%more available symbols in equations
\usepackage{verbatim}
\usepackage{natbib}				% control bibliography
	\setcitestyle{aysep={},citesep={;},notesep={:}}		% set citation style: no separator between author and year, separator between multiple citations = semicolon, separator between citation and postnote (e.g. page number) = colon
\usepackage{scalefnt}			% arbitrary scaling for font size

\usepackage{abstract}
\renewcommand{\absnamepos}{flushleft}
\setlength{\absleftindent}{0pt}
\setlength{\absrightindent}{0pt}

\newcommand{\glosssize}{\scalefont{.85}}			% second line of gloss small
	\let\eachwordtwo=\glosssize						
\newcommand{\gender}{\textunderscore}			% separator for genderwashing
\newcommand{\source}[1]{\hfill (#1)\\[-0.2cm]}	% sources for examples
\newcommand{\sourcewrap}[1]{\\ \strut \hfill (#1)\\[-0.2cm]}	% sources for examples
\newcommand{\abrac}[1]{$\langle$#1$\rangle$}		% ⟨angle brackets⟩ with \abrac{}
\renewcommand{\labelitemi}{$\bullet$}			% change symbol used in lists
\renewcommand{\labelitemii}{$\circ$}				% change symbol used in lists

\pagenumbering{gobble}
\renewcommand{\section}[2]{}

\begin{document}

\begin{center}

%\vspace*{3cm} % for prescribed page margins

%	\includegraphics[height=17mm]{grafiken/uzh_logo_d_pos.pdf}\\[0.5cm] % this height is fixed by the university's corporate design rules
	\Large{%Universität Zürich\\
	\bfseries Readings in Neroinformatics} \\[1cm] 

%	{\huge \bfseries titel:\\
% untertitel}\\[3.5cm]

%	\Large{BA-Arbeit}\\[0.1cm]
%	\Large{Hauptfach (90 ECTS)} \\[0.1cm]
%	\Large{Vorgelegt im Juni 2015} \\[1.5cm]

	\Large{ Ephraim Seidenberg}\\[0.1cm]
	10-931-798\\[0.1cm]
%	\Large{ \bfseries Betreuung:}\\[5cm]
	
	\vfill

\end{center}

\vspace{1cm}

\bibliographystyle{unified-caps}
\bibliography{absbiblio}
\nocite{hodgkin1952quantitative}

\begin{abstract}
\normalsize
When excited, the surface membrane of a nerve fibre carries electric current. Giving rise to complicated phenomena such as the action potential and refractory period, this has been associated with changes in in- and outward movement of ions. For a precise understanding, a mathematical description of these ion movement changes in relation to a given membrane potential is necessary. Our work shows that the quantitative behaviour of a model nerve can be predicted under a variety of conditions and that the responses to electrical stimuli can be explained by reversible alterations in sodium and potassium permeability arising from the changes in membrane potential. To this end, equations and parameters were fitted to experimental curves from previously obtained voltage clamp data. Agreement with fair accuracy was found for electrical properties of the squid giant axon such as form, duration and amplitude of both 'membrane' and propagated spike, conduction velocity, impedance changes during spike, refractory period, ionic exchanges, subthreshold responses and oscillations. In addition, many of the phenomena of excitation, including anode break excitation and accommodation can at least qualitatively be accounted for. These observations provide a solid basis for refined models of membrane function and examination of higher-level processes in nerve.

%The voltage clamp data obtained previously are used to find equations
%which describe the changes in sodium and potassium conductance associated
%with an alteration of membrane potential. The parameters in these equations
%were determined by fitting solutions to the experimental curves relating
%sodium or potassium conductance to time at various membrane potentials.


%2. The equations, given on pp. 518-19, were used to predict the quantitative
%behaviour of a model nerve under a variety of conditions which corresponded
%to those in actual experiments. Good agreement was obtained in the following
%cases:
%(a) The form, amplitude and threshold of an action potential under zero
%membrane current at two temperatures.
%(b) The form, amplitude and velocity of a propagated action potential.
%(c) The form and amplitude of the impedance changes associated with an
%action potential.
%(d) The total inward movement of sodium ions and the total outward
%movement of potassium ions associated with an impulse.
%(e) The threshold and response during the refractory period.
%(f) The existence and form of subthreshold responses.
%(g) The existence and form of an anode break response.
%(h) The properties of the subthreshold oscillations seen in cephalopod axons.
%3. The theory also predicts that a direct current will not excite if it rises
%sufficiently slowly.
%4. Of the minor defects the only one for which there is no fairly simple
%explanation is that the calculated exchange of potassium ions is higher than
%that found in Sepia axons.
%5. It is concluded that the responses of an isolated giant axon of Lr5ligo to
%electrical stimuli are due to reversible alterations in sodium and potassium
%permeability arising from changes in membrane potential.

%18 grad, 21 grad in choline seawater
\end{abstract}
\end{document}