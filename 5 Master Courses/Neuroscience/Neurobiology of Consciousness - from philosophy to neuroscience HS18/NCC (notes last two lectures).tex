\documentclass{article}

\begin{document}
	\title {Notes of last three lectures of Neurobiology of Consciousness}
	\maketitle
	
	\section{Agnosia, Epilepsy, Sleep Walking and DID - 6.12.2018}
	\textbf{Def. Agnosia: }Absence of knowledge. Types: akinetopsia, achromatopsia, Capgras syndrome.\\
	Capgras syndrome: loss of emotional connection to between sensory signals and the associated emotions. Patients believe their family, friends etc. are not the real friends, since they do not have the normal emotions to them associated anymore. 
	\paragraph{Patient DF.} Low-level vision quite intact. She can detect flashes, navigate, catch and throw things and follow things with eyes. But she cannot interpret 2D lines and cannot recognize them (apperceptive agnosia). She cannot name or copy an object if it is 2D. If she is asked to draw something from memory, she can do it, but cannot recognize later what she drew.\\
	In an experiment, she was unable to judge the orientation of a bar. On the other hand, if she had to put a letter through an opening with different orientations, she performed almost as good as controls.\\
	$\Rightarrow$ interpreting impossible, but guiding possible.\\
	$\Rightarrow$ there seems to be a visual pathway which is conscious (ventral stream) and another which is unconscious (dorsal stream).
	
	\paragraph{Epilsepy.}Re-occurrence of seizures periodically $\Rightarrow$ temporal brain dysfunction (e.g. motor dysfunction due to seizure).\\
	$\Rightarrow$ loss of consciousness and in involvement of whole cortex (if generalized seizure - \textit{grand mal})\\
	There are also seizures that leave the patient awake, but later on, they will not be able to recall any of the events (people can interact with such a person, such as asking facts or making him perform a motor task).\\
	\textbf{Split-brain patients:} Corpus callosum severed. Left hemisphere has the language center, which can speak and respond to question. Right hemisphere can make a drawing of word. $\Rightarrow$ two minds in one body = two persons in one body. These patients have an impairment in short-term memory as well as a small impairment in concentration (become fatigued faster). 
	
	\paragraph{Sleep walking.}Occurs during non-REM sleep. Eyes can be open, but sleepwalker will have no conscious recollection of the events that happened during a sleepwalking episode.
	
	\paragraph{DID.}2 or more distinct identities or personality states, where each of them has their own set of experiences. There are often two states: the neutral identity state and the traumatic identity state. The NIS has no access to traumatic memory and allows normal day functioning. Furthermore, their psychobiological make-up is different, meaning that their EEG, brain blood flow, heart rate, blood pressure etc. were different.
	
	\paragraph{Coma patients.}Some coma patients are capable of imagining doing an activity when asked to do and their fMRI patterns are similar to controls ($\Rightarrow$ not all coma patients are basically brain dead).
	
	\section{Hallucinations - 13.12.2018 (Student Presentation)}
	\textbf{Defining hallucinations:} Sensory perceptions which disagree with those of other observers in the same setting\footnote{Book: A dictionary of hallucinations.}. A percept out of sync.\\
	This definition is rather democratic, because reality/hallucinations are defined by what most people observe vs. what a minority observes.\newline
	
	\paragraph{Charles Bonnet Syndrome.}Visual hallucinations in people who have lost some or all of their sight.\\
	Sensory deprivation tanks remove sensory input and lead to unusual mental phenotypes in humans.\\
	The brain is used to process information all the time. In case of intense sensory information deprivation, it simply starts to create its own data to process.\newline
	
	\textbf{Real sensory ghosts:} If the brain constructs a body model based on evidence, can we swap bodies by giving alternative evidence? This way, the body can adopt even a piece of wood as thinking its own, when both the wood and the hand (not seen by subject) are stroked.
	
	\section{Last lecture - 20.12.2018}
	Discussion:\\
	1) Will there be a satisfactory answer regarding the nature of consciousness?\\
	2) If yes: in which direction is one supposed to go to find a satisfactory answer?\\
	3) Can we build conscious artificial intelligences/machines?\\
	
\end{document}